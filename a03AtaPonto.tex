%
% brownletter_example.tex - an example latex file to illustrate brownletter.cls
%
% Copyright 2003, Nesime Tatbul (tatbul@cs.brown.edu)
%
% Copyright 2021, (classes para substituto UFCG) Sergio Rivero (sergio.rivero@ufcg.edu.br)
%

\documentclass[12pt]{uaefata}

\usepackage{hyperref}


\date{} % if no date specified, today's date is used 

%\subject{Ata} % optional subject line



\newcommand{\cdta}{Huguinho Duck }

\newcommand{\cdtb}{Luizinho Duck }

\newcommand{\cdtc}{Zezinho Duck }


%arquivo das notas didaticas dos candidatos

\newcommand{\cdtand}{41,9 }
\newcommand{\cdtbnd}{42,1 }
\newcommand{\cdtcnd}{42,0 }

\newcommand{\cdtant}{41,9}
\newcommand{\cdtbnt}{42,1}
\newcommand{\cdtcnt}{42,0}

\newcommand{\cdtanto}{41,9}
\newcommand{\cdtbnto}{42,1}
\newcommand{\cdtcnto}{42,0}


% Arquivo das notas finais dos candidatos

\newcommand{\cdtanf }{41,9 }
\newcommand{\cdtbnf}{42,1 }
\newcommand{\cdtcnf}{42,0 }


%%% Nomes

\newcommand{\prsd}{Profa. Dra. Uhura }
\newcommand{\prsdtit}{Presidente }
\newcommand{\mba}{Prof. Dr. Data }
\newcommand{\mbb}{Spock }
\newcommand{\mbt}{Membro }


\signature{\prsd \\ \prsdtit \\
	\vspace{40pt}
	\mba \\ \mbt \\
	\vspace{40pt}
	\mbb \\ \mbt}



\newcommand{\edital}{Edital número 42 de 11 de Março de 1952 }
\newcommand{\resconc}{Resolução número HAL de 12 de Janeiro de 1992 }
\newcommand{\cargo}{Presidente da Galáxia }
\newcommand{\concurso}{Concurso Interestelar }



\begin{document}

\begin{letter}{
		\textbf{Ata do sorteio do ponto e da ordem de apresentação dos candidatos do \concurso, para \cargo,  a que se refere o \edital.}
	}
 
\opening{}

Às 8:00h do dia 06 de dezembro de 2021 a banca examinadora do \concurso do \edital reuniu-se com os candidatos inscritos no concurso para a seleção do ponto e da ordem de apresentação da prova didática do referido concurso. Para a execução da reunião usou-se a sala virtual do aplicativo starlink trek. A seleção do ponto da prova didática se deu utilizando a ferramenta de geração de números aleatórios existente na página de internet do endereço numpy.random. O ponto selecionado aleatoriamente foi o número 42 (\textit{Propulsão de Dobdra teoria de Alcuberre.}) e a ordem de apresentação dos candidatos sorteada foram os números \textit{13, 21, 34}. Os nomes dos candidatos na ordem de apresentação seguem abaixo.

\centering

\begin{tabular}{|l|c|}
	\hline
	\textbf{Candidata(o)} & \textbf{Número}\\
	\hline
	\cdta & 34\\
	\cdtb & 21 \\
	\cdtc & 13 \\
	\hline
\end{tabular} 



\closing{Campina Grande, \today.}



%\cc{J. Kirschenbaum}

\end{letter}

\end{document}
