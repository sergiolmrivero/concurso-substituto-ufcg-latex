%
% brownletter_example.tex - an example latex file to illustrate brownletter.cls
%
% Copyright 2003, Nesime Tatbul (tatbul@cs.brown.edu)
%
% Copyright 2021, (classes para substituto UFCG) Sergio Rivero (sergio.rivero@ufcg.edu.br)
%

\documentclass[12pt]{uaefata}

\usepackage{hyperref}

\date{} % if no date specified, today's date is used 

%\subject{Ata} % optional subject line



\newcommand{\cdta}{Huguinho Duck }

\newcommand{\cdtb}{Luizinho Duck }

\newcommand{\cdtc}{Zezinho Duck }


%arquivo das notas didaticas dos candidatos

\newcommand{\cdtand}{41,9 }
\newcommand{\cdtbnd}{42,1 }
\newcommand{\cdtcnd}{42,0 }

\newcommand{\cdtant}{41,9}
\newcommand{\cdtbnt}{42,1}
\newcommand{\cdtcnt}{42,0}

\newcommand{\cdtanto}{41,9}
\newcommand{\cdtbnto}{42,1}
\newcommand{\cdtcnto}{42,0}


% Arquivo das notas finais dos candidatos

\newcommand{\cdtanf }{41,9 }
\newcommand{\cdtbnf}{42,1 }
\newcommand{\cdtcnf}{42,0 }


%%% Nomes

\newcommand{\prsd}{Profa. Dra. Uhura }
\newcommand{\prsdtit}{Presidente }
\newcommand{\mba}{Prof. Dr. Data }
\newcommand{\mbb}{Spock }
\newcommand{\mbt}{Membro }


\signature{\prsd \\ \prsdtit \\
	\vspace{40pt}
	\mba \\ \mbt \\
	\vspace{40pt}
	\mbb \\ \mbt}



\newcommand{\edital}{Edital número 42 de 11 de Março de 1952 }
\newcommand{\resconc}{Resolução número HAL de 12 de Janeiro de 1992 }
\newcommand{\cargo}{Presidente da Galáxia }
\newcommand{\concurso}{Concurso Interestelar }




\begin{document}

\begin{letter}{
		\textbf{Comunicado sobre o sorteio do ponto e da ordem de apresentação dos candidatos do \concurso, para \cargo,  a que se refere o \edital.}
	}
 
\opening{Prezadas(os) Candidatas(os)}

A banca examinadora do \concurso do \edital comunica a vossas senhorias os detalhes das apresentações da prova didática. 
\begin{enumerate}
	\item O ponto sorteado foi o número 42 (\textit{Propulsão de Dobdra teoria de Alcuberre.})
	\item As(Os) candidatas(os) deverão estar disponíveis uma hora antes do horário determinado para sua prova didática (item 6.4 do edital)
	\item A apresentação da prova didática é regida pela resolução \resconc do conselho universitário da UFCG
	\item As salas para as apresentações são as seguintes:
		\begin{itemize}
			\item \today - Manhã - \href{https://meetInSpace.spc}{https://meetInSpace.spc}
		\end{itemize}
	\item Não é permitido aos candidatos assistir as provas de outros, isto implica que apenas um candidato deverá estar na sala, no momento de sua apresentação.
	\item A ordem  e horários das provas didáticas seguem abaixo
\end{enumerate}
\centering

\begin{tabular}{|l|c|}
	\hline
	\textbf{Candidata(o)} & \textbf{Dia e Horário}\\
	\hline
	\cdta & \today - 8:00h\\
	\cdtb & \today - 9:00h \\
	\cdtc & \today - 10:00h \\
	\hline
\end{tabular} 



\closing{Campina Grande, \today.}



%\cc{J. Kirschenbaum}

\end{letter}

\end{document}
