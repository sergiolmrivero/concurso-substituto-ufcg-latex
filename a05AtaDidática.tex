%
% brownletter_example.tex - an example latex file to illustrate brownletter.cls
%
% Copyright 2003, Nesime Tatbul (tatbul@cs.brown.edu)
%
% Copyright 2021, (classes para substituto UFCG) Sergio Rivero (sergio.rivero@ufcg.edu.br)
%
\documentclass[12pt]{uaefata}

\usepackage{hyperref}


\date{} % if no date specified, today's date is used 

%\subject{Ata} % optional subject line



\newcommand{\cdta}{Huguinho Duck }

\newcommand{\cdtb}{Luizinho Duck }

\newcommand{\cdtc}{Zezinho Duck }


%arquivo das notas didaticas dos candidatos

\newcommand{\cdtand}{41,9 }
\newcommand{\cdtbnd}{42,1 }
\newcommand{\cdtcnd}{42,0 }

\newcommand{\cdtant}{41,9}
\newcommand{\cdtbnt}{42,1}
\newcommand{\cdtcnt}{42,0}

\newcommand{\cdtanto}{41,9}
\newcommand{\cdtbnto}{42,1}
\newcommand{\cdtcnto}{42,0}


% Arquivo das notas finais dos candidatos

\newcommand{\cdtanf }{41,9 }
\newcommand{\cdtbnf}{42,1 }
\newcommand{\cdtcnf}{42,0 }


%%% Nomes

\newcommand{\prsd}{Profa. Dra. Uhura }
\newcommand{\prsdtit}{Presidente }
\newcommand{\mba}{Prof. Dr. Data }
\newcommand{\mbb}{Spock }
\newcommand{\mbt}{Membro }


\signature{\prsd \\ \prsdtit \\
	\vspace{40pt}
	\mba \\ \mbt \\
	\vspace{40pt}
	\mbb \\ \mbt}



\newcommand{\edital}{Edital número 42 de 11 de Março de 1952 }
\newcommand{\resconc}{Resolução número HAL de 12 de Janeiro de 1992 }
\newcommand{\cargo}{Presidente da Galáxia }
\newcommand{\concurso}{Concurso Interestelar }



\begin{document}

\begin{letter}{
		\textbf{Ata da Avaliação da Prova Didática dos candidatos do \concurso para \cargo  a que se refere o \edital.}
	}
 
\opening{}

Às 14:00h do dia 09 de dezembro de 2021 a banca examinadora do \concurso do \edital reuniu-se para a avaliação dos resultados da prova didática do referido concurso. As planilhas de notas para as provas de cada candidato para cada avaliador foram apresentadas. As referidas planilhas continham notas de 0 a 100 dos de cada candidato participante da prova didática com a média ponderada de acordo com os critérios estabelecidos no artigo 22 da resolução 01/2018-UFCG. Calculou-se a média aritmética simples das notas dos candidatos dadas pelos avaliadores, que seguem abaixo. Os avaliadores estão nominados na tabela de notas da seguinte maneira:
\begin{enumerate}
	\item \prsd
	\item \mba
	\item \mbb
\end{enumerate}


\newpage


\centering


\textbf{Média da prova didática e notas  por avaliador }

\begin{tabular}{|l|c|c|c|c|}
		\hline
		\textbf{Candidata(o)}	&	Avaliador 1	&	Avaliador 2	&	Avaliador 3	&	\textbf{Nota}		\\
		\hline										
		\cdta	&	\cdtand	&	\cdtand	&	\cdtand	&	\cdtand \\
		\cdtb	&	\cdtbnd	&	\cdtbnd	&	\cdtbnd	&	\cdtbnd	\\
		\cdtc	&	\cdtcnd	&	\cdtcnd	&	\cdtcnd	&	\cdtcnd	\\
		\hline										

\end{tabular}

\closing{Campina Grande, \today.}



%\cc{J. Kirschenbaum}

\end{letter}

\end{document}
