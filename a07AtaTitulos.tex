%
% brownletter_example.tex - an example latex file to illustrate brownletter.cls
%
% Copyright 2003, Nesime Tatbul (tatbul@cs.brown.edu)
%
% Copyright 2021, (classes para substituto UFCG) Sergio Rivero (sergio.rivero@ufcg.edu.br)
%

\documentclass[12pt]{uaefata}

\usepackage{hyperref}

%\title{ModeloUAEF}
%

%\date{} % if no date specified, today's date is used 

%\subject{Ata} % optional subject line





\newcommand{\cdta}{Huguinho Duck }

\newcommand{\cdtb}{Luizinho Duck }

\newcommand{\cdtc}{Zezinho Duck }


%arquivo das notas didaticas dos candidatos

\newcommand{\cdtand}{41,9 }
\newcommand{\cdtbnd}{42,1 }
\newcommand{\cdtcnd}{42,0 }

\newcommand{\cdtant}{41,9}
\newcommand{\cdtbnt}{42,1}
\newcommand{\cdtcnt}{42,0}

\newcommand{\cdtanto}{41,9}
\newcommand{\cdtbnto}{42,1}
\newcommand{\cdtcnto}{42,0}


% Arquivo das notas finais dos candidatos

\newcommand{\cdtanf }{41,9 }
\newcommand{\cdtbnf}{42,1 }
\newcommand{\cdtcnf}{42,0 }


%%% Nomes

\newcommand{\prsd}{Profa. Dra. Uhura }
\newcommand{\prsdtit}{Presidente }
\newcommand{\mba}{Prof. Dr. Data }
\newcommand{\mbb}{Spock }
\newcommand{\mbt}{Membro }


\signature{\prsd \\ \prsdtit \\
	\vspace{40pt}
	\mba \\ \mbt \\
	\vspace{40pt}
	\mbb \\ \mbt}



\newcommand{\edital}{Edital número 42 de 11 de Março de 1952 }
\newcommand{\resconc}{Resolução número HAL de 12 de Janeiro de 1992 }
\newcommand{\cargo}{Presidente da Galáxia }
\newcommand{\concurso}{Concurso Interestelar }


\begin{document}

\begin{letter}{
		\textbf{Ata da Avaliação da Prova de Títulos dos candidatos do \concurso para \cargo  a que se refere o \edital.}
	}
 
\opening{}

Às 11:00h do dia 10 de dezembro de 2034 a banca examinadora do\concurso do \edital reuniu-se para a avaliação dos resultados da Prova de Títulos do referido concurso. Avaliou-se, seguindo o artigo 18 da resolução \resconc, apenas os candidatos que obtiveram nota maior ou igual a 70 na prova de títulos. De posse dos currículos e das referidas comprovações, a banca examinadora procedeu o cômputo dos pontos de títulos dos candidatos, seguindo a tabela do anexo II da \resconc, bem como os artigos da seção II da referida resolução e os artigos 25 a 28 da \resconc. As planilhas de notas para as provas de cada candidato foram compiladas e procedeu-se o ajuste dos pontos de cada candidato de acordo com o artigo 28 da \resconc, que seguem apresentados na tabela abaixo.


\newpage


\centering

\textbf{Somatória e Nota Ajustada da Prova de Títulos}

\begin{tabular}{|l|r|r|}
	\hline
	\textbf{Candidato}	&	\textbf{Nota}	&	\textbf{Ajustada}	\\
	\hline
	\cdta	&	\cdtanto	&	\cdtant	\\
	\cdtb	&	\cdtbnto	&	\cdtbnt	\\
	\cdtc	&	\cdtbnto	&	\cdtcnt	\\
	\hline
\end{tabular}




\closing{Campina Grande, 10 de dezembro de 2021.}



%\cc{J. Kirschenbaum}

\end{letter}

\end{document}
